% !TEX program = xelatex

\documentclass{resume}
\usepackage{xltxtra,fontspec,xunicode}
\usepackage[slantfont,boldfont]{xeCJK} % 允许斜体和粗体
\usepackage{tabularx}
%\usepackage{zh_CN-Adobefonts_external} % Simplified Chinese Support using external fonts (./fonts/zh_CN-Adobe/)
\usepackage{zh_CN_fonts_internal} % Simplified Chinese Support using system fonts

\usepackage{titlesec}

% Spacing corresponding to `book` class
% \titlespacing*{\section} {0pt}{3.5ex plus 1ex minus .2ex}{2.3ex plus .2ex}
% \titlespacing*{\subsection} {0pt}{3.25ex plus 1ex minus .2ex}{1.5ex plus .2ex}

% Spacing before/after reduced by 1ex each
\titlespacing*{\section} {0pt}{0ex plus 1ex minus .2ex}{1.3ex plus .2ex}

\begin{document}
	\pagenumbering{gobble} % suppress displaying page number
	
	\name{李颖喆}
	
	\basicInfo{
		\phone{(+86) 131-6139-8723}
		\email{lyz0326@126.com}
		\faUser{ 24岁}
		%\faBook\ \hyperref[http://blog.csdn.net/qwop446]{\underline{http://blog.csdn.net/qwop446}}}
		}
	\basicInfo{
		\linkedin\ \hyperref[https://www.linkedin.com/in/liyingzhe]{\underline{https://www.linkedin.com/in/liyingzhe}}
		%\faBook\ \hyperref[http://blog.csdn.net/qwop446]{\underline{http://blog.csdn.net/qwop446}}
		}
		
	\section{\faCogs\ 专业技能}
	\begin{itemize}[parsep=0.5ex]
		\item 3年Java开发经验,扎实的数据结构和算法功底,对多线程技术、异步、并发有深入理解
		\item 对Mysql数据库的基本理论和内部实现机制有比较深刻的理解,能够熟练运用
		\item 了解Redis的设计与实现,熟悉应用常用数据结构应用场景及其数据库使用
		\item 熟悉消息中间件的应用场景,深入了解RocketMQ技术内幕,了解其高可用机制
		\item 具有分布式开发经验,了解项目开发常用分布式框架如Redisson、LCN、XXL-JOB
		\item 深入理解SSM等开源框架设计原理,熟练使用SpringBoot、SpringCloud相关技术栈
		\item 掌握Freemarker、H5、JavaScript、Vue等前端技术,熟练使用多种前端框架
		\item 具有基本阅读英文文档及查阅英文资料的能力
	\end{itemize}
	
	\section{\faGraduationCap\ 教育经历}

	\begin{tabularx}{\textwidth}{@{}X X X r@{}}
		\textbf{太原科技大学}  & \textbf{   华科学院     } & \textbf{ 计算机科学与技术 }  & 2013.09 -- 2017.06  \\
		\multicolumn{4}{@{}l}{{本科} }
	\end{tabularx}
		
	
	\section{\faUsers\ 工作经历}
	
	\begin{tabularx}{\textwidth}{@{}X X X r@{}}
		\textbf{杭州海仓科技} & \textbf{       跨境仓储} & \textbf{杭州}   & 2019.05 -- Presen   \\
	\end{tabularx}
	
		中国(杭州)跨境电子商务综合试验区首批试点企业、浙江省“十二五”电子商务百强及浙江电商跨境10强企业、国家高新技术企业、省跨境电商区域性重点服务企业、第一批省级供应链创新与应用试点企业

	\begin{tabularx}{\textwidth}{@{}X X X r@{}}
		\textbf{遍我游网络科技}  & \textbf{         旅游服务}  & \textbf{  北京} & 2017.03 -- 2019.04\\
	\end{tabularx}

		苏州市文化广电和旅游局主导,苏州全域旅游项目。依托互联网、移动互联网、大数据,实现苏州旅游六要素+"一二三产业"大整合,为游客提供串联式、高品质的旅游产品及服务;对旅游产业进行监管,并实现旅游产业及服务的不断升级

	% Reference Test
	%\datedsubsection{\textbf{Paper Title\cite{zaharia2012resilient}}}{May. 2015}
	%An xxx optimized for xxx\cite{verma2015large}
	%\begin{itemize}
	%  \item main contribution
	%\end{itemize}
	
	
	\section{\faGraduationCap\ 项目经历}
	
	\begin{tabularx}{\textwidth}{@{}X X r@{}}
		\textbf{跨境通关服务平台} & \textbf{海仓科技} & 2019.05 -- Present \\
	\end{tabularx}
	跨境通关服务平台通过前置接口,将淘宝、天猫、京东、苏宁及其他自营平台的商家订单接入平台,为商家提供三单申报、物流适配、仓储推送、智能调度和数据分析服务,并且辅以支付单代推、加签和系统管理服务,让企业能实时查看订单的申报、通关和物流进度
	\begin{itemize}
		\item 引入Nacos作为服务注册配置中心,改用MySQL进行持久化,提取共享配置文件统一管理
		\item 负责Gateway相关开发,如路由、限流和基于Oauth2的权限验证
		\item 使用Spring Cloud Stream做为消息处理,集成RocketMQ并应用于如订单服务、物流服务、申报服务间的服务调用
		\item 研究并落地规则引擎Drools的应用,通过Docker搭建BRMS实现规则在线编辑、在线打包、远程执行,使用于商家订单导入,仓库订单对接等
		\item  使用Mybatis-plus底层代码自动生成,Redisson分布式事务锁,并封装成公共组件提高开发质量与效率
		\item 协助运维进行Jenkins相关配置如指定Nacos、JVM参数,以及为Vue前台项目和Gateway进行Nginx路由配置
	\end{itemize}
	
	\begin{tabularx}{\textwidth}{@{}X X r@{}}
		\textbf{智能仓储管理系统} & \textbf{海仓科技} & 2019.05 -- Present \\
	\end{tabularx}
	海仓智能WMS系统覆盖仓储物流全部业务,全自动化智能导向、提高工作效率、 库位精确定位管理、状态全面监控、实时掌控库存情况、合理保持和控制企业库存。 覆盖了所有库内操作环节,包括了收货、 上架、移库、盘点、波次、拣货、包装、复核、称重等,完美支持电商、零售、生鲜、第三方物流等多行业仓储管理需求
	\begin{itemize}
	\item 与产品部门对接相关需求及可行性讨论,与第三方公司如网易考拉、菜鸟、慧仓、音飞进行技术方案制定
	\item 将原有SpringMVC项目重构为SpringBoot,使用AOP对服务对接权限验证、日志信息进行统一处理,使用策略模式实现不同的服务调用
	\item 通过XXL-JOB进行配置化任务调度,提供多种任务调度策略,如补货的黑盒模式和白盒模式
	\item 使用Redis的setnx和Lua脚本通过自定义注解的方式实现分布式事务锁,使用String数据结构进行设备信息共享,使用Redis的list实现任务调度
	\item 通过Dubbo服务进行仓库作业各系统间的交互,打通由理货入库、补货、拣选、播种、打包复核等整体仓库作业流程
	\item 使智能调度平台产品化,实现硬件系统与仓库系统配置化接入,为实现多方接入打下基础
	\end{itemize}
	
	
	\begin{tabularx}{\textwidth}{@{}X X r@{}}
		\textbf{苏州好行} & \textbf{遍我游} & 2017.03 -- 2019.04 \\
	\end{tabularx}
	苏州好行为来苏州游客提供票务、特产、交通、攻略等一站式旅游服务的APP
	\begin{itemize}
		\item 使用Spring、SpringMVC、Mybatis搭建项目,使用Freemarker模版引擎制作web页面
		\item 使用AdminLTE搭建后台管理系统,实现对产品、订单、攻略、会员、销售的设计实现
		\item 通过CMS内容管理系统对APP首页进行动态化配置发布,以展示不同的主题风格
		\item 使用Memcached对商品订单等数据进行缓存,使用Redis实现Session共享
		\item 负责与微信支付宝进行支付对接、智游宝平台进行票务对接、以及天猫、总入口等分销商的对接
		\item 通过扫描二维码绑定微信openId的方式实现销售的统计,并在高峰期对性能进行查询优化
		\item 使用百度地图API实现旅游巴士的实时定位与展示已经公交线路的查询功能
		\item 通过nginx的upstream实现服务负载均衡,负责日常项目的迭代部署
	\end{itemize}
	
	
	\begin{tabularx}{\textwidth}{@{}X X r@{}}
		\textbf{游急便} & \textbf{遍我游} & 2017.05 -- 2019.04 \\
	\end{tabularx}
	游急便为旅游途中的游客提供精准定位,提供一个在陌生城市生活、外出、旅行时帮助用户提 供找厕所的服务的工具
	\begin{itemize}
		\item 使用AdminLTE+Freemarker重构后台管理系统
		\item 通过Jquery实现多个厕所进行信息属性对比操作的功能
		\item 通过微信扫描生成的厕所二维码进行评价,并实现向优质评价发送微信红包的功能
		\item 接入百度地图API实现厕所的定位与线路规划,并提供厕所的相应信息
		\item 实现景区管理员、志愿者微信绑定,扫描厕所二维码对后台定制的巡查报表进行打分填报统计的功能
	\end{itemize}
	
	\begin{tabularx}{\textwidth}{@{}X X r@{}}
		\textbf{苏州旅游数据中心} & \textbf{遍我游} & 2017.10 -- 2019.04 \\
	\end{tabularx}
	数据中心对苏州市全域范围内的资源整合,统一进行管理,实现苏州全市、区县、乡镇、景区等全覆盖,“政府引导+企业主导”相结合,对旅游产业进行监管
	\begin{itemize}
		\item 使用SSM+LigerUI+Freemarker完成数据中心的搭建
		\item 利用JdbcTemplate+Mysql实现页面的动态化配置,如查询条件、结果、编辑页面、保存操作等
		\item 使用Mysql读写分离进行数据处理,Redis做数据缓存
		\item 利用LigerUI的动态表单设计,以JSON的格式完成报表填报定制化的实现
		\item 通过配置SQL实现各个页面的查询功能,并对SQL进行优化
		\item 通过Cookie获取登录信息,通过SQL配置映射实现与其他系统的对接
		\item 将系统部署于政府云内网,利用Jsonp解决与其他系统交互的跨域问题
	\end{itemize}
	
	\begin{tabularx}{\textwidth}{@{}X X r@{}}
		\textbf{totem软件平台} & \textbf{遍我游} & 2017.09 -- 2017.11 \\
	\end{tabularx}
	totem软件平台是随需而变的业务环境所需的敏捷IT基础设施,为企业信息化建设提供灵活、快捷、可信赖的平台。以数据架构为切入点,从统一展现、专业系统、配置环境、数据管理,在快速开发平台的基础上通过服务保障为企业提供全方位的IT技术服务
	\begin{itemize}
	 	\item 使用SSM+EasyUI+Freemarker实现需求分析
		\item 通过Mysql等数据库反向生成相应表,实现原型设计的功能
		 \item 使用Freemarker编写模版实现通用SSM后台模版、各种风格前台代码生成的功能
	\end{itemize}

	
%	\begin{tabularx}{\textwidth}{@{}X X X r@{}}
%		\textbf{深圳大学 (SZU)} & \textbf{数学与统计学院} & \textbf{应用数学} & 2011.9 -- 2015.6 \\
%		\multicolumn{4}{@{}l}{{本科} GPA: 3.8/4,校级一等、特等奖学金}
%	\end{tabularx}
	
		
	%% Reference
	%\newpage
	%\bibliographystyle{IEEETran}
	%\bibliography{mycite}
\end{document}
